%% This is an example first chapter.  You should put chapter/appendix that you
%% write into a separate file, and add a line \include{yourfilename} to
%% main.tex, where `yourfilename.tex' is the name of the chapter/appendix file.
%% You can process specific files by typing their names in at the 
%% \files=
%% prompt when you run the file main.tex through LaTeX.
\chapter{Implementation}\label{intro-ch}

\section{Spark}

All of the code was implemented in Spark. Although the code was implemented in Spark, 
it could also be implemented in MapReduce to achieve similar performance improvement.
The  Resilient Distributed Dataset(RDD) is the main interface within Spark.
The RDD can be created from data or from another RDD. The key attributes of an RDD are its inputs,
the number of partitions, and how each of its partitions is computed based on its inputs.

Profressor Matei Zaharia added code that allowed the tracking of sizes of map output partitions.   

\section{ShuffledRDD}

The RDD we developed is a new version of ShuffledRDD, ShuffledRDD2. 
Its inputs are first a shuffle dependency, which is basically a bunch of map output partitions, and second, a number of reducers, or equivalentally the 
number of partitions for ShuffledRDD2. 
In the regular ShuffledRDD, each of its partition naively requests a segment of map output partitions as depicted in Figure~\ref{fig:shuffle_unbalanced}.
ShuffledRDD2 implements the more complicated scheme seen in Figure~\ref{fig:shuffle_balanced.} 
The current Spark API only allows reducer partitions to request consecutive map partitions. In other words, it is impossible for a ShuffledRDD2 partition to have map output partitions 1 and 3, without having 2. For this constraint and the given number of reducer partitions , ShuffleRDD2 assigns map output partitions to optimally balance its reduce partitions. 

\section{Joins}

\subsection{ShuffleReader Changes}

As mentioned in the broadcast join section, the bigger RDD must stay in place. The current interface only allows a reducer to request a specific map output partition
 from all of the mappers. For the bigger RDD, the system would then have to request map output partitions from other machines, which defeats the purpose of the broadcast join. Thus, we added the capability of requesting a specific partition from just one mapper. 

\subsection{ShuffleJoinRDD and BroadcastJoin RDD}

We implement two different type of RDDs, the ShuffledJoinRDD and the
BroadcastJoinRDD. Both of these RDD's take two shuffle dependencies, which remember
are basically the outputs of map stages, partitioned in a certain way. These 
dependencies must be partitioned in the same way; otherwise, two identical keys would not map to the same partition index.  


The ShuffleJoinRDD implementation is very similar to ShuffledRDD. Instead of fetching map output partitions
from just one dependency, it fetches the corresponding map output partitions from both dependencies.   
The user specificies the number of ShuffleJoinRDD partitions and each paritition request a corresponding fraction
of the map output partitions.
For instance, ShuffledJoinRDD partition 1 will fetch Dataset1 Partition 1 and Dataset2
Partition 1 from all of the workers. Once these partitions are fetched, its create a map with the key-value pairs of the smaller partition.
It iterate through the keys of the bigger partition, seeing if they are present in this map, and if so,  adding the intersection to the ourput.\\

The BroadcastJoinRDD implements the broadcast shuffle.
For each BroadcasstJoinRDD partition, it requests one local map output partition from the bigger RDD using the new request capability and  
all of the paritions from the smaller RDD. The number of partitions is equal to the number of partitions of the bigger input RDD. The system use the same strategy
with the map and the iteration as the ShuffledJoinRDD to then find the intersections.

\subsection{Joins in Spark SQL}

Many programmers and data analysts prefer not to use the RDD interface
and are more familiar with the sql; thus, Spark offers a sql like interface or SparkSql. One popular operation within sql is join.
Although the user still writes in sql, Spark still executes the code using RDDs.

Because we are not just using the RDD interface and Spark automatically converts the sql query into a query plan,
the implementation is much more complicated and thus we only implement our optimization for sort merge join.
Although the exact semantics for how a sort merge join can be found here,the sort merge join still uses the shuffle and requires that every
key that could intersect should be sent to the same reducer.
To help achieve this, the sort merge join applies an exchange operator on each of the mapoutputs. These exchange operators produce
ShuffleRowRDDS, which for our purposes are equivalent to ShuffledRDDS. In the next stage, each partition in the first ShuffledRowRDD is compared
to the partition with the same index in the second ShuffledRowRDD. The only difference between this and how the join RDDs work is pretty semantic
in  that instead of one RDD requesting partitions from multiple mappers, two RDD's repartition their data and then another one compared partition by partion.
By default, the code performs a shuffle join almost exactly in a manner with how the ShuffleJoinRDD works. One ShuffleRowRDD requests 
the corresponding partitions from its mapoutput just like Figure~\ref{fig:shuffle_unbalanced} and the other ShuffleRowRDD does the exact same but with its dataset.  

However, if only one input RDD is smaller then a user-configured threshold, the system uses the broadcast join optimization. The bigger ShuffledRowRDD will be exactly like its
parent. The other ShuffledRowRDD will have the same number of partitions as the bigger ShuffledRowRDD with each partition containing the entirety of the smaller input RDD. 
The correctness guarentees is the same as the BroadcastJoinRDDs.  

