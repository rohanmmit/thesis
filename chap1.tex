%% This is an example first chapter.  You should put chapter/appendix that you
%% write into a separate file, and add a line \include{yourfilename} to
%% main.tex, where `yourfilename.tex' is the name of the chapter/appendix file.
%% You can process specific files by typing their names in at the 
%% \files=
%% prompt when you run the file main.tex through LaTeX.
\chapter{Introduction}\label{intro-ch}

A number of applications require a domain expert to visually inspect and
process a dataset on a stream of incoming data. The problem with such manual
inspection is it's inability to scale as datasets grow exponentially
\cite{exp-growth}. As the data set grows, it becomes difficult to visualize and
interact with \cite{immens} and there are also cases of false positives
which the expert should not have to manually reclassify. We with to focus on
the field of medical data, where doctors have to view a patient's data and
extract relevant information for treatment. Specifically, we focus on
electroencephalogram (EEG) readings, a test which is used to detect
abnormalities related to the electrical activity of the brain. \\

Today, doctors are faced with having to store large amounts of patient data
without analysis since they lack the tools to efficiently view datasets are a
large scale. To address this issue, we have designed and implemented a system,
Pinky, which can process large amounts of EEG data, allowing near real-time
interaction for analysis.


\section{Pinky}

Pinky is a doctor's newest tool for analysing the brain, see
Figure~\ref{fig:pinky-and-the-brain}. Working with a team of researchers at
Massachusetts General Hospital (MGH), we have designed and implemented the
system to handle the fast growing corpus of EEG data that is being collected.
This end-to-end system handles the storage, computation, and visualization of
EEG data.  The goal of the system is to provide a scalable architecture for
concurrent analysis of patient records with near real time interactivity. Each
layer of the system has been optimized for use and evaluated across hundreds of
GB of patient data. \\

\begin{figure}[h]
\begin{center}
\includegraphics[scale=0.5]{./img/pinky-and-the-brain.png}
\caption{Etymology of Pinky's name.}
\label{fig:pinky-and-the-brain}
\end{center}
\end{figure}

\section{Overview of EEG Analysis}

Seizures are defined as transient aberrations in the brain's electrical
activity. People with the central nervious system disorder, epilepsy, suffer
from recurrent seizures, often happening suddenly at unpredictable times.  A
seizure can vary from a lapse of attention to a whole-body convulsion.
Frequent seizures are dangerous, as they can increase risk of sustaining
physical injuries or even result in death \cite{eeg-ml}. \\

A method for detecting the onset of epileptic seizures is done by analysing the
scalp EEG, a non-invasive measure of the brain's electrical activity.
Continuous EEG (cEEG) data is typically recorded using 19 silver/silver
chloride electrodes, affixed to the scalp according to the international 10-20
system \cite{ceeg-1}. Figure~\ref{fig:electrodes} shows a drawing of the
placement of sensors on a patient's scalp. \\

\begin{figure}[h]
\begin{center}
\includegraphics[scale=0.5]{./img/electrodes.png}
\caption{Electrode placement on a patient's scalp.}
\label{fig:electrodes}
\end{center}
\end{figure}

EEG recordings typically take place over a 24-hour period and must be reviewed
and screen by trained individuals such as attending physicians,
epilepsy/neurophysiology fellows, or registered EEG technicians
(encephalographers) \cite{ceeg-3}. Unlike traditional epilepsy monitoring units
which focus on provoking and capturing seizures, the goal of cEEG studies is to
rapidly and efficiently identify seizures for future prevention. This leads to
an increase in the number of cEEG recordings for preventative measures, and
intensive care unit centers become overwhelmed with the analysis of the growing
dataset and small number of trained individuals. Methods to screen long EEG
recordings without sacrificing accuracy are necessary to be able to process all
of the data. \\

Typically, EEGs displays no more than 10-15 seconds of data per screen of raw
voltage readings and requires an analyst to simultaneously inspection multiple
channels. On the contrary, a compressed spectral array \cite{csa} or
spectrogram display may show 2 to 8 hours of data on a single color
map\cite{ceeg-3}. This allows analysts to quickly screen long periods of EEG
data, determining which segments, if any, require direct review. Spectrogram
review reduces cEEG review time by $78\%$ \cite{ceeg-2}, with minimal loss of
sensitivity compared with conventional review.\\

Spectrograms are the most widely used compressed data format for EEG data
\cite{ceeg-1}. A spectrogram consists of three-dimensional plots with time on
the x-axis, frequency on the y-axis, and EEG power on the z-axis. An analyst
typically views four spectrograms concurrently, mapped to different regions of
the brain. Each region is formed by using multiple EEG channels where an EEG
channel is formed by taking the difference between voltages measured at two
electrodes. This captures the summed potential of millions of neurons
\cite{eeg-ml}. Figure~\ref{fig:electrodes} shows the electrode placement on
the patient's scalp, yielding four regions for analysis: left lateral power,
\c{LL}, (Fp1-F7, F7-T3, T3-T5, T5-O1), left parasagittal power, \c{LP},
(Fp1-F3, F3-C3, C3-P3, P3-O1), right lateral power, \c{RL}, (Fp2-F8, F8-T4,
T4–T6, T6-O2), right parasagittal power, \c{RP}, (Fp1-F4, F3-C4, C4-P4, P4-O2).
\\

As the corpus of data collected as ICU continues to grow, efficient mechanisms
to store and visualize this data efficiently is key for analysts to quickly
view patient screenings. Pinky aims to provide this for analysts, giving them a
simple yet powerful interface to quickly view spectrogram data.

\section{System Architecture}

The system is comprised of three coupled layers which handle, storage,
computation and visualization. Figure~\ref{fig:system-architecture} shows the
overall architecture of the system.

\begin{figure}[h]
\begin{center}
\includegraphics[scale=0.75]{./img/system-architecture.png}
\caption{Pinky System Architecture.}
\label{fig:system-architecture}
\end{center}
\end{figure}

\subsection{Storage Layer}

The storage layer, discuss in detail in Chapter~\ref{storage-ch}, is
responsible for storing raw EEG patient data and a copy of the calculated
spectrogram. This datastore must be optimized for both reads and writes of
array based data for multidimensional arrays on the order to tens to hundreds
of GB.

\subsection{Compute Layer}

The compute layer, discuss in detail in Chapter~\ref{compute-ch}, is meant to
be an extendible module which handles the algorithms to calculate the
spectrogram and other EEG related calculations. As we discuss in
Section~\ref{discuss-ch:future-work}, there are a number of extensions the
project can take, thus it is important that an interested developer can easily
add functionality to this layer. In addition, the compute layer contains two
servers, one to server the array based data, interfacing with the optimized EEG
algorithms, and a lightweight server for the web resources to be served to the
visualization layer.

\subsection{Visualization Layer}

The visualization layer, discussed in detail in Chapter~\ref{viz-ch}, is a browser
based module that renders the data to the client. The interface allows users to
query based on a patient's id (medical record number, \c{mrn}) and view a spectrogram
for a given time interval.

\subsection{Visgoth System}

Since enabling interactivity is an important design criteria, we have designed
and built a optimization module for time series visualization in the browser,
Visgoth. This is discussed in detail in Chapter~\ref{visgoth-ch}.

\section{Usage}

The project code base is available publicly on Github \cite{github}
\url{https://github.com/joshblum/eeg-toolkit} with documentation for installing
the project for development. In addition, we have created Docker \cite{docker}
images that can easily be install for production use. Armed with a dataset, any
curious doctor should be able to install the images and load the data for
analysis. The docker images are available for public use here:
\url{https://hub.docker.com/r/joshblum/eeg-toolkit-webapp} and
\url{https://hub.docker.com/r/joshblum/eeg-toolkit-toolkit}. Specific
installation instructions can be found with the github project.

\section{Contributions}

Pinky makes the following contributions:

\begin{itemize}

  \item Implements an abstraction for array based storage systems.

  \item Implements 4 different backends to adhere to the abstraction.

  \item Evaluates the different backends for multiple input ranges and
    workloads.

  \item Implements optimized algorithms for analysing EEG data.

  \item Provides an extendible framework for accessing array based data and
    visualizing it in the browser.

  \item Implements scalable in-browser visualizations using the client's GPU.

  \item Implements a new system, Visgoth, for reducing latency for time series
    based visualizations.

\end{itemize}

These contributions enable doctors and medical expert analysts to interactively
analyse EEG data at scale.

