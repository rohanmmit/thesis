\chapter{Compute Layer}\label{compute-ch}

The compute layer is responsible for several important sections of Pinky's
functionality. First, this layer handles interaction between the storage layer
by using the \c{StorageBackend} abstraction to read and write dataset and
visualization layer by performing calculations a client's request and
transferring data back for rendering. In addition, this layer also contains
implementations of optimized algorithms for processing EEG data to produce the
spectrogram visualization.

\section{Design}

Since this layer of the system supports a variety of features, we separate the
design points for each. As with other layers, minimizing runtime was one of the
key design goals, in addition to enabling developers to easily modify and add
new processing algorithms.

\subsection{Websocket Server}

The websocket server is responsible for transferring array based data to the
client for rendering. The server must be able to quickly serialize data from
the storage layer, namely multidimensional arrays of floating point values. The
server must be able to run concurrently in order to handle multiple client
requests, processing the requests in parallel to reduce latency.

\subsection{Webapp Server}

The webapp server is responsible for serving web resources such as JavaScript,
HTML, and CSS to the client. The scripts that are sent to the client are
responsible for communicating with the websocket server, retrieving data for
display. This server is separate from the websocket server to simplify the
implementation since the webapp server is not subject to the same performance
requirements when transferring datasets over the network. The separation of
these two servers allows for modules to be easily added to the webapp framework
using existing frameworks (see \ref{compute-ch:implementation-webapp}) such as
authentication.

\subsection{File Ingestion Daemon}

The file ingestion daemon is responsible for collecting new files that are
added to the system and converting them for use. This makes use of conversions
between the EDF format and the required format for the \c{StorageBackend} that
is in use. Input files are initially received in the EDF format and must
subsequently be converted without affecting other clients who are using the
system.

\subsection{EEG Spectrogram}\label{compute-ch:design-spectrogram}

TODO, discussing the algorithms in detail.

\section{Implementation}

The compute layer is implemented almost entirely in C++ except for the file
ingestion daemon and webapp server which are written in Python. The Python
implementations comprise a much smaller part of the compute layer codebase
since there are many libraries available in the community. Although there is a
cost for using multiple languages within a system, Python seemed the
appropriate choice in this circumstance due to the reduced developer time and
dependency on mature libraries. C++ was chosen for the other parts of the
module since the performance gains with a compiled language were necessary. C++
was chosen over pure C for the available libraries for linear algebra
processing \cite{arma} and network communication \cite{websocket-server}.

\subsection{Websocket Server}\label{compute-ch:implementation-ws-server}

The websocket server primarily depends on the open source library
\cite{websocket-server}. This library implements the websocket protocol and
provides a simple interface for transferring data across the network. The
library was not without problems, over the course of use we reported and helped
debug several concurrency issues with the generous help of the maintainer.
These bug fixes also resulted in large performance gains from the initial
implementation with reduced network latency. \\

When a client loads the Pinky frontend into their browser, the client sends a
request to the websocket server to open a connection. When the client wishes to
browse a patient's data, a request is send specifying the patient's \c{mrn},
the \c{channel} (LL, LP, RL, RP, see
\ref{compute-ch:implementation-spectrogram}), a \c{start\_time} and \c{end\_time}
given in hours. \\

The websocket server then requests a spectrogram with the given values either
computing the spectrogram on the fly or using a precomputed spectrogram
calculation. The websocket server with potentially downsample the response to
meet client latency requirements, however, this functionality is detailed in
chapter \ref{visgoth-ch}. \\

To send data back to the client, a JSON encoded header is created, notifying
the client on the calculation results. The header contains a small amount of
metadata such as the sampling rate, \c{fs}, calculated \c{start\_time} and
\c{end\_time} (these values are validated to the bounds of the dataset),
spectrogram matrix dimensions and the \c{channel} the calculation was completed
for. The messaging protocal transfer binary data encoding a header length in a
\c{uint32\_t} followed by the serialized header information and the matrix data.
All messages are byte aligned to 8 bytes so a small amount of padding may be
added for performance reasons. \\

The websocket spawns $n_{processors} - 1$ threads to serve requests with. For
development and testing we have use the CSAIL OpenStack framework to create
multicore virtual machines for testing.

\subsection{Webapp Server}\label{compute-ch:implementation-webapp}

The webapp server is implemented using the Python mircoframework Flask
\cite{flask}. The Flask framework allows simple serving of HTML webpages with
support of the Jinja2 \cite{jinja2} templating system. This is vastly easier to
develop on than maintain than a comparable C++ implementation, and since
performance is not an issue it is a viable option. \\

The server provides web resources for the client as well as a few pages with
information about the project.

\subsection{File Ingestion Daemon}

The file ingestion daemon is implemented as a Python script that monitors the
server's filesystem for changes in order to ingest data files as they are
added. Taking advantage of the Watchdog \cite{watchdog} library, the script is
able to receive events for changes to the filesystem for a given folder. Upon
receiving a notification, the script will convert the file to the necessary
format and also precompute the spectrogram. To process the file, the script
calls the command line programs \c{edf\_converter <mrn>}
\ref{storage-ch:implementation-cmd} and \c{precompute\_spectrogram <mrn>}
\ref{compute-ch:implementation-cmd}. \\

This functionality allows an administrator to dump EDF files onto the
filesystem of the server and allow them to automatically be queried by an
analyst.

\subsection{EEG Spectrogram}\label{compute-ch:implementation-spectrogram}

The spectrogram calculation involves reading portions of the raw EEG data,
taking the FFT of the data with a sliding window in time and storing the
results in a matrix for visualization. The way the spectrogram is computed can
vary based on several parameters, and to simplify the implementation, we store
all relevant spectrogram parameters in an object, \c{SpecParams}. When a client
requests a spectrogram for a given region of the brain, this object is created,
building the parameters from the input \c{mrn}, \c{start\_time} and
\c{end\_time}. \\

The \c{SpecParams} object contains the following parameters, the comments next
to each parameter describe it's use.

\begin{lstlisting}
    string mrn; // patient medical record number
    StorageBackend* backend; // array storage backend
    float start_time; // start time of the spectrogram
    float end_time; // end time of spectrogram
    int start_offset; // start offset of raw data
    int end_offset; // end offset of raw data
    int spec_start_offset; // start offset of spectrogram data
    int spec_end_offset; // start offset of spectrogram data
    int fs; // sample rate
    int nfft; // number of samples for fft
    int nstep; // number of steps
    int shift; // shift size for windows
    int nsamples; // number of samples in the spectrogram
    int nblocks; // number of blocks
    int nfreqs; // number of frequencies
\end{lstlisting}

Using the \c{SpecParams} object, we can calculate the spectrogram for the EEG
data. The algorithm's pseudocode is presented below. A detailed description of
the algorithm can be found in \ref{compute-ch:design-spectrogram}.

\begin{lstlisting}
void eeg_spectrogram(SpecParams* spec_params, int ch, fmat& spec_mat)
{
  // Get the column which contains the first channel for the region.
  ch_idx1 = DIFFERENCE_PAIRS[ch].ch_idx[0];
  frowvec vec1, vec2; // initialize read vectors
  read_array(mrn, ch_idx1, start_offset, end_offset, vec1);

  for (int i = 1; i < NUM_DIFFS; i++)
  {
    // Get the column which contains the next channel for the region.
    ch_idx2 = DIFFERENCE_PAIRS[ch].ch_idx[i];
    read_array(mrn, ch_idx2, start_offset, end_offset, vec2);
    // take the difference between the channel pair
    frowvec diff = vec2 - vec1;

    // fill in the spec matrix with FFT values
    FFT(spec_params, diff, spec_mat);
    swap(vec1, vec2);
  }
  spec_mat /=  (NUM_DIFFS - 1); // average diff spectrograms
  spec_mat = spec_mat.t(); // transpose the output
}
\end{lstlisting}

The definition of the constants \c{DIFFERENCE\_PAIRS} and \c{NUM\_DIFFS} are
omitted from the pseudocode for simplicity. The \c{DIFFERENCE\_PAIRS} simply
defines which channels to take differences to form a region (see
\ref{compute-ch:design-spectrogram}) and \c{NUM\_DIFFS=4} since we compute the
spectrogram across four different regions of the brain. \\

The FFT algorithm is implemented using the FFTW library \cite{fftw} for optimal
performance and uses a Hanning windowing function between FFT runs. We make use
of the Armadillio C++ linear algebra library \cite{arma} for simplifying
vector/matrix calculations.

The Hamming window is defined as follows:

\begin{lstlisting}
void hamming(int windowLength, float* buf)
{
  for (int i = 0; i < windowLength; i++)
  {
    buf[i] = 0.54 - (0.46 * cos(2 * M_PI *
                    (i / ((windowLength - 1) * 1.0))));
  }
}
\end{lstlisting}

\subsection{Command Line Programs}\label{compute-ch:implementation-cmd}

The compute module offers two command line scripts, \c{test} and
\c{precompute\_spectrogram <mrn>}. The \c{test} script is used for testing new
functionality to an algorithm or \c{StorageBackend}. The
\c{precompute\_spectrogram} program will take a \c{mrn} as input and calculate
and store the spectrogram for the given \c{mrn}.

\subsection{Optimizations}

The implementation of the \c{eeg\_spectrogram} algorithm was designed with
minimizing memory consumption in mind. For this reason the \c{vec1}, \c{vec2},
and \c{spec\_mat} buffers were reused during the calculation. We considered
computing each of the differences for a region in parallel, however computing
each region in parallel (parallelized at the websocket server) was performant
enough. Previous iterations involved serializing the output of the Armadillio
matrix (\c{fmat}), however we found that instead we could directly access a
pointer of the raw buffer memory. This optimization help significantly since it
eliminated a memory allocation and data copying before sending over the
network. \\

Another minor optimization is the use of the \c{static inline} keyword. We use
this for several helper functions to reduce function call overhead. In
addition, we always pass Aramdillo objects by reference and not value, avoiding
a copy on function calls.

\section{Related Work}

